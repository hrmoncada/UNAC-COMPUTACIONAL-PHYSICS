\documentclass[a4paper,12pt]{exam}

\usepackage[utf8]{inputenc} % Para caracteres acentuados
\usepackage{amsmath}        % Paquete para matemáticas avanzadas
\usepackage{amssymb}        % Símbolos matemáticos
\usepackage{graphicx}       % Para insertar gráficos
\usepackage{multicol}       % Para varias columnas de preguntas
\usepackage{geometry}       % Para ajustar los márgenes

\geometry{top=1in,bottom=1in,left=1in,right=1in}

\title{Práctica Calificada}
\author{Curso de Cálculo}
\date{}

\begin{document}

\maketitle

\begin{center}
    Nombre: \underline{\hspace{8cm}} \\
    Fecha: \underline{\hspace{8cm}}
\end{center}

\vspace{0.5cm}

\begin{questions}

    % Pregunta 1: Derivada
    \question[10] Calcule la derivada de las siguientes funciones:
    \begin{parts}
        \part \( f(x) = 3x^5 - 2x^3 + x - 7 \)
        \part \( g(x) = \frac{2x^2 + 3}{x^3 - 1} \)
    \end{parts}
    \vspace{2cm}

    % Pregunta 2: Derivada Implícita
    \question[10] Utilice derivación implícita para encontrar \(\frac{dy}{dx}\) si:
    \[
        x^2 + xy + y^2 = 10
    \]
    \vspace{2cm}

    % Pregunta 3: Razón de Cambio
    \question[10] En un cierto momento, la sombra de un poste de 5 metros de altura es de 12 metros. Si la longitud de la sombra está disminuyendo a una tasa de 0.5 metros por segundo, ¿a qué velocidad se está moviendo la punta de la sombra?
    \vspace{3cm}

    % Pregunta 4: Extremos Absolutos (Criterio de la Primera Derivada)
    \question[10] Encuentre los extremos absolutos de la función \( f(x) = x^3 - 6x^2 + 9x + 15 \) en el intervalo \( [0, 5] \), utilizando el criterio de la primera derivada.
    \vspace{3cm}

    % Pregunta 5: Extremos Absolutos (Criterio de la Segunda Derivada)
    \question[10] Determine los máximos y mínimos relativos de la función \( g(x) = x^4 - 4x^3 + 6x^2 - 24x \) utilizando el criterio de la segunda derivada.
    \vspace{3cm}

\end{questions}

\end{document}
